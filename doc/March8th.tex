\documentclass[11pt]{beamer}
\usetheme{Warsaw}
\usepackage[utf8]{inputenc}
\usepackage[english]{babel}
\usepackage{enumitem}
\usepackage{mathtools}
\usepackage{amsmath}
\usepackage{amsfonts}
\usepackage{amssymb}
\usepackage{graphicx}
\author{Aravind and Jeel}
\title{IDP - EE3025}
\setbeamercovered{transparent}
\setbeamertemplate{navigation symbols}{}
%\logo{} 
\institute{IIT Hyderabad} 
\date{8th March, 2019}
%\subject{} 

\begin{document}

\begin{frame}
\titlepage
\end{frame}

\begin{frame}{Project Idea}
	\textsf{Sobel Filter For Edge Detection on Images}
	% write about sobel filter
	% math equations
\end{frame}

\begin{frame}{Implementation Plan}
	\begin{itemize}
		\item[•] {Load image into FPGA
			\begin{itemize}
				\item[•] Convert image to hex file using \textit{ffmpeg}
				\item[•] Stream these pixel values as bytes from Raspberry Pi to FPGA RAM using \textit{ffmpeg}
				\item[•] Use coordinate decoder module on sequence of pixels (bytes). For this we need to specify the image size. Later we plan to extend to use pilot bytes to detect row endings. 
			\end{itemize}
		}
		\pause
		\item[•] { To Implement an efficient Sobel Filter on FPGA
			\begin{itemize}
				\item[•] (write about sobel filter implementation in verilog)
			\end{itemize}
		}
	\end{itemize}

\end{frame}

\begin{frame}{Loading Image into FPGA using Raspberry Pi}
write about SPI interface \\
maybe about expected progress by next presentation
\end{frame}

\end{document}
